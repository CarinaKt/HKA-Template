

\section{Einleitende Worte}
% label dient zum referenzieren
\label{sec:einleitende-worte}

\todo{Test2}

\begin{figure}
  \centering
  \includegraphics[width=0.1\textwidth]{figures/HKA_Logo.png}
  \caption{Ein Bild}
\end{figure}
Das Bild zeigt unser Logo\footnote{Bitte korrekt verwenden.}.


Hallo das ist ein Test

mal sehen ob oder sonderzeichen

Funktionieren äöüß \parencite[Vgl.][9-26]{Tibi2011}

\begin{center}
  \includegraphics[width=0.1\textwidth]{figures/HKA_Logo_Lila.png}
  \captionof{figure}
      [Nicht gleitende Abbildung im Verzeichnis]
      {Eine nicht gleitende Beispielabbildung}
\end{center}

\section{Fortgeschrittene Anwendung}
\label{sec:fortg-anwend}

\subsection{Was macht Alice im Wunderland?}
\label{sec:was-macht-alice}

In Abschnitt \ref{sec:einleitende-worte} wurde ein Mädchen namens
Alice erwähnt. Was sie im Wunderland erlebt, kann in einem Buch
nachgelesen werden.

\subsection{Analyse}
\label{sec:analyse}

\begin{center}
  \includegraphics[width=0.1\textwidth]{figures/HKA_Logo_Lila.png}
  \captionof{figure}
      [Nicht gleitende Abbildung im Verzeichnis]
      {Eine nicht gleitende Beispielabbildung}
\end{center}


Die Gleichungen \eqref{eq:1} bis \eqref{eq:3} beherrschen wir bestens.
Alice, von der wir auf Seite \pageref{sec:einleitende-worte} gehört
haben, kennt diese Gleichungen wahrscheinlich nicht.

\begin{itemize}
  \item Alice im Wunderland
  \item Till Eulenspiegel
\end{itemize}

\begin{itemize}
  \item Alice im Wunderland
  \item Till Eulenspiegel
  \item Harry Potter
        \begin{itemize}
          \item Der Stein der Weisen
          \item Kammer des Schreckens
          \item Der Gefangene von Askaban
          \item Der Feuerkelch
          \item Der Orden des Phönix
        \end{itemize}
  \item Jim Knopf
\end{itemize}

\begin{enumerate}
  \item Der Stein der Weisen
  \item Kammer des Schreckens
  \item Der Gefangene von Askaban
  \item Der Feuerkelch
  \item Der Orden des Phönix
  \item \ref{sec:einleitende-worte}
\end{enumerate}

\section{Mathematik}
\label{sec:mathematik}

\subsection{Unterstufe}
\label{sec:unterstufe}

\begin{equation*}
  a + 2 = c
\end{equation*}

\begin{equation*}
  a_{ij} - a_2 = 0
\end{equation*}

\begin{equation*}
  \frac{1}{a} + \frac{1}{b} = \frac{a+b}{ab}
\end{equation*}

\begin{equation*}
  \sigma + \tau = \alpha
\end{equation*}

\begin{equation*}
  \int_{}^{}  \,dx
\end{equation*}

\begin{equation*}
  \sum_{n = 1}^{\infty}
\end{equation*}

\subsection{Oberstufe}
\label{sec:oberstufe}

\begin{equation}
  \label{eq:1}
  \left( \frac{a}{b} \right)' = \frac{a'b-ab'}{b^{2}}
\end{equation}

Es gilt die Invariante $b \neq 0$.

\begin{equation}
  \label{eq:2}
  \int\limits_{a}^{b} x^{2} \, dx = \frac{ b^{3} - a^{3} }{3}
\end{equation}

\begin{equation}
  \label{eq:3}
  c = \sqrt{ a^{2} + b^{2} }
\end{equation}

\lstset{language=Pascal}

\begin{lstlisting}[caption=Pascal Code, label=lst:mycode]
  begin
    a := 3;
    b := a * 4;
    c := (b + a)/ 2
  end.
  \end{lstlisting}

Mit dem Anchor-Tag (\lstinline!<a href="">...</a>!) lassen...

\begin{tabular}{rlr}
  Position & Beschreibung & Anzahl \\
  1        & Lenkrad      & 1      \\
  2        & Reifen       & 4      \\
  3        & Motor        & 1      \\
\end{tabular}

\begin{equation*}\boxed{
    \begin{gathered}
      a^2 + b^2 = c^2 \\
      \sigma + \alpha = \beta
    \end{gathered}}
\end{equation*}