
%\maketitle
\section{One-Pager Thesis} 
\subsection*{Carina Kühnert, 73683\\ \today{}, Malsch}

% Thema in deutsch und englisch
Ziel dieser Arbeit ist die Erweiterung einer mobilen Roboterplattform, sodass auf nicht statische Hindernisse reagiert werden kann.
%%Extending an existing sensor system to deal with non-static obstacles.
A mobile robot platform is extended to react to non-static obstacles in this work.

% Umfeld/ Arbeitsbedingungen
Die Bachelorarbeit wird in der Firma Aunovis GmbH geschrieben und ist Teil der Innovationsprojekte des Unternehmens.
Das Ziel ist die Weiterentwicklung eines bestehenden Prototyps.
%Dort wird an einem Werbeprojekt gearbeitet. 
%Es soll später auf Messen zu Werbezwecken eingesetzt werden.
Später soll der Prototyp auch auf Messen zu Werbezwecken eingesetzt werden.

Hauptsächlich wird der Schwerpunkt der Arbeit sich auf den Umgang mit nicht statischen Hindernissen wie 
beispielsweise einem Menschen, der durch die Bahn des Roboters läuft, konzentrieren.

% Aktueller Stand/ vorhanden Sensorik
Die Arbeit schließt an eine vorangegangene Arbeit an, 
in dieser wurde eine mobile Roboterplattform, die ihre Umgebung kartiert, aufgebaut.
Diese verfügt über RealSense D435 (Tiefenkamera), eine T265 (Tracking-Kamera) und einen Nvidia Jetson AGX Computer
und hat ROS2 als Middleware installiert. Ebenfalls wurde die ODrive Motorsteuerung (Python-Bibliothek) in ROS2 integriert.
Aktuell kann der Roboter erkennen, wenn etwas in seine berechnete Bahn kommt. Er wartet, bis die Bahn wieder frei ist. 
Aufgrund der Kameraposition kann die Umgebung bei engen Kurven um ein Hindernis erst ab einem Abstand von 73 cm beobachtet werden. Kreuzt etwas Kleines, z.B. 
ein Baby, die Roboterbahn innerhalb dieser Entfernung, so wird dies nicht wahrgenommen. 

% Anforderungen
Der Roboter hat das Aussehen des R2D2 Roboters, so ist bei der Auswahl ergänzender Sensoren darauf zu achten, 
dass das Aussehen weiterhin dem R2D2 entspricht. Ebenfalls sollten die ergänzten Funktionen möglichst ressourcenschonend sein, 
um die Batterielaufzeit nicht unnötig zu verkürzen. 
%%Zusätzliche Sensoren können ergänzt werden.
Ebenfalls sind sicherheitstechnische Betrachtungen zu berücksichtigen, um Schäden an Menschen oder der Umgebung zu vermeiden.
Hier soll der Schwerpunkt nicht auf der genauen Umsetzung der ISO Normen liegen.

% Literatur
Als Literatur steht die vorangegangene Bachelorarbeit, sowie die Dokumentationen von ROS2 und den verwendeten Sensoren zu Verfügung.

% Arbeitsschritte
Mögliche Features:
\begin{itemize}
    \item anhalten, wenn sich etwas im Weg befindet,
          was sich nicht in der Punktewolke befindet (z.B. ein vorbeilaufender Mensch)
    \item Gesichtserkennung, um Personen zu erkennen und personalisiert reagieren zu können
    \item Zuordnung des Gesichts zu einer Platzreservierung in einem Buchungssystem
    \item (erkannte) Person zu einem bestimmten Punkt führen
    \item überprüfen, ob erkannte Person noch folgt
    \item folgen einer Person auf ein Signal hin
    %\item komfortablere Steuerung der Roboterplattform
\end{itemize}